\documentclass[letter paper, 11pt]{article}
\usepackage[utf8]{inputenc}
\usepackage{fullpage} % changes the margin
\usepackage[T1]{fontenc}
\usepackage{selinput}
\usepackage{enumitem}
\usepackage{listings}
\usepackage{amsmath}
\usepackage{amssymb}
\usepackage{setspace}
\usepackage{bm}
\usepackage{mathtools}

\title{Introduction to Real Analysis by Bartle and Sherbert}
\author{Quinn Murphey}
\date{Fall 2019}

\begin{document}

\maketitle
\setcounter{section}{4}
\section{Continuity}
\subsection{Continuous Functions}
\begin{itemize}
    \item[1.] This proof is a simple application of Theorem 4.1.8 (The Sequential Criterion for Limits).
    
    \begin{itemize}
        \item[($\Rightarrow$)] Assume that f is continuous at point $c\in A$. This means that the $\lim_{x\rightarrow c}f(x) = f(c)$. Then, by Theorem 4.1.8, every sequence $(x_n)$, if $(x_n)$ converges to $c$, then $f(x_n)$ converges to $f(c)$.
        
        \item[($\Leftarrow$)] Assume that for every sequence that converges to $c$, $f(x_n)$ converges to $f(c)$. Then, by 4.1.8, $\lim_{x\rightarrow c} f(x) = f(c)$. Which means that $f$ is continuous at $c$. To handle the special case $x_n=c$ for some $m$-tail of $(x_n)$ not covered in 4.1.8, we can easily see that $f(x_n)$ becomes a constant sequence $f(c)$, therefore $\lim_{x\rightarrow c} f(x) = f(c)$.
    \end{itemize}
    
    \item[3.] 
    
    \item[4.]
    \begin{itemize}
        \item[(a.)] Let $f(x)=[[x]]$. Then $f(x)$ is constant in intervals $[n,n+1)$ for $n\in\mathbb{N}$. Therefore constant on that interval. However at each $n\in\mathbb{N}$, $f$ isn't continuous because no matter the $\delta>0$, the $\epsilon\geq 1$ because $[[n-\delta]]$ has a lower value than $[[n]$ for any $\delta>0$.
        
        \item[(b.)] Let $f(x)= x[[x]]$. The argument is exactly the same as above, except instead of $[n,n+1)$ being a constant function, it is a linear function with slope $[[x]]$. And at $n$, there are jump discontinuities of height $|[[x]]|$ (absolute value of greatest integer function).
        
        
    \end{itemize}
    
    \item[5.] Since $\lim_{x\rightarrow 2}f(x) = 5$, then we can make $f$ continuous by letting $f(2)=5$ because then $f(2) = \lim_{x\rightarrow 2}f(x)$.
    
    \item[7.] Let $f(c) = C$, then we have $C > 0$. If we let $0<\epsilon<C$ then define $\delta_\epsilon$ as any real such that $|x-c|<\delta_\epsilon \Rightarrow |f(x)-f(c)|<\epsilon$. Then, by definition for any $x\in V_\delta(c)$, $f(x)$ is greater than $0$.
    
    \item[11.] If we let $\delta = \varepsilon/K$, then obviously, $|x-y|<\delta \Rightarrow K|x-y|<\varepsilon$. Then by transitivity, $|f(x)-f(y)|<\varepsilon$.
    
    \item[12.] Due to the rationals being dense in the reals, if any real number is not mapped to 0, there does not exist a delta $>0$ such that there is not a rational in the delta neighborhood, therefore, in order for $f$ to be continuous, it must equal 0 at all points.
    
    \item[13.] The only points at which $g$ is constant is when $\lim_{x\rightarrow c}2x=\lim_{x\rightarrow c}x+3$. Since each of these functions are continuous we can remove the limits and just solve for $2x=x+3$ which is only $x=3$. 
    

\end{itemize}

\subsection{Combinations of Continuous Functions}
\begin{itemize}
    \item[1.] 
    
    \item[3.]
    
    \item[5.]
    
    \item[6.]
    
    \item[10.]
    
    \item[12.]
    
    \item[13.]
    
\end{itemize}

\subsection{Continuous Functions on Intervals}
\begin{itemize}
    \item[1.] 
    
    \item[3.]
    
    \item[5.]
    
    \item[6.]
    
    \item[7.]
    
    \item[8.]
    
    \item[10.]
    
    \item[13.]
    
    \item[15.]
    
\end{itemize}

\subsection{Uniform Continuity}
\begin{itemize}
    \item[1.] 
    
    \item[2.]
    
    \item[3.]
    
    \item[6.]
    
    \item[7.]
    
    \item[8.]
    
    \item[11.]
        
    \item[12.]
    
    \item[15.]
    
\end{itemize}

\subsection{Continuity and Gauges}
\begin{itemize}
    \item[1.] 
    
    \item[2.]
    
    \item[4.]
    
    \item[6.]
    
    \item[7.]
    
    \item[9.]
    
\end{itemize}

\subsection{Monotone and Inverse Functions}
\begin{itemize}
    \item[1.] This a simple application of the definition of absolute minimum and increasing function. For $f:[a,b]\rightarrow\mathbb{R}$ to be increasing, $x,y\in[a,b], x<y\Rightarrow f(x)\leq g(x)$. Therefore, since $a\leq x$ for $x\in[a,b]$, $f(a)$ is a absolute minimum occurring at $x=a$. Replacing the $\leq$ with $<$ in our definition of increasing function, we can see that if $f$ is strictly increasing, then $a$ is the only point such that $f(x)=f(a)$ therefore an absolute minimum. 
    
    The case for $b$ being an (unique) absolute maximum is the exact same application of the definitions of absolute max and increasing functions and the interval $[a,b]$.
    
    \item[2.] $f+g$ is obviously an increasing function since for arbitrary $x,y\in I$, such that $x<y$ we have $f(x)\leq f(y)$ and $g(x)\leq g(y)$. Therefore adding these inequalities together we get $$f(x)+g(x) \leq f(y)+g(y) \Rightarrow$$
    $$(f+g)(x) \leq (f+g)(y)$$ for all $x<y$ in $I$. Therefore $f+g$ is increasing.
    
    Replacing all $\leq$ with $<$ above we can easily see that the same is true for strictly increasing functions adding to create a new strictly increasing function.
    
    \item[4.] Let $x,y$ be chosen like above. Then we have $f(x)\leq f(y)$ and $g$ similar. Since $g(a)\geq0$ for all $a\in I$, we can multiply the two inequalities together to get
    $$f(x)g(x)\leq f(y)g(y) \Rightarrow$$
    $$(fg)(x)\leq (fg)(y)$$
    for all $x<y$ in $I$. Therefore $fg$ is an increasing function.
    
    \item[5.] 
    \begin{itemize}
        \item[$(\Rightarrow)$] Assume $f:I\rightarrow\mathbb{R}$ is continuous at $a$ and increasing with $I=[a,b]$. Then assume $f(a)\not=\inf\{f(x):x\in(a,b]\}$. Obviously $f(a)$ is a lower bound of this set due to the increasing criteria. Therefore the only way for $f(a)$ not be the infimum is if there exists a strictly greater lower bound $\alpha>f(a)$. Then for $\epsilon<\alpha-f(a)$, there exists no delta neighborhood of $x$ such that $|f(a)-f(x)|<\epsilon$ so $f$ is not continuous at $a$. This is a contradiction, therefore $f(a)=\inf\{f(x):x\in(a,b]\}$
        
        \item[$(\Leftarrow)$] Assume that $f$ is increasing and $f(a)=\inf\{f(x):x\in(a,b]\}$. Then, for any $\epsilon>0$, there exists an $c\in I$ such that $f(c)<f(a)+\epsilon$. Which implies for all $x<c$, this is also true. Therefore, letting $\delta=c-a$, we have $|x-c|<\delta$ implies that $0<f(x)-f(a)<\epsilon$ so $|f(x)-f(a)|<\epsilon|$. Therefore $f$ is continuous at $a$.
        
    \end{itemize}
    
    \item[7.] Since $f$ is increasing, $\lim_{x\rightarrow c^-}f(x)=\sup\{f(x):x<c\}$ and similar for $\lim_{x\rightarrow c^+}f(x)=\inf\{f(x):x>c\}$. Therefore by the subtractive properties of the infimum and supremum. $\lim_{x\rightarrow c^+}f(x)-\lim_{x\rightarrow c^-}f(x) = \inf\{f(x):x>c\}- \sup\{f(x):x<c\} = \inf\{f(x)-f(y):x<c,c<y,x,y\in I\})$ which is our solution.
    
    \item[10.] Assume $f:[a,b]\rightarrow\mathbb{R}$ has an absolute maximum at $c<b$, then for all $x$ such that $c<x<b$, $f(x)\leq f(c)$. If any $x$ satisfies $f(x)=f(c)$, then clearly $f$ is not injective. Therefore we need only check the case that $f(x) < f(c)$. Then we may also assume that $f(a)<f(c)$ for a similar argument. Then due to continuity, there exists an $x$ such that (*)$f(a)<f(x)<f(c)$, which by bolzano's intermediate value theorem, there exists a $y<c$ such that $f(x)=f(y)$ for all $x$ satisfying (*) which is nonempty. Therefore $f$ is not injective.
    
    \item[12.] Assume that $f$ is injective and not strictly increasing. Then there exists some $1>f(x)\geq f(y)$ for $x<y$. Then by intermediate value theorem, there exists a $c>y$ such that $f(c)=f(x)$, which is a contradiction. Therefore $f$ is strictly increasing.
     
\end{itemize}

\section{Differentiation}

\subsection{The Derivative}

\begin{itemize}
    \item[1.]
    \begin{itemize}
        \item[a)]
        
            Let $f(x) = x^3$. Then $$f'(c) = \lim_{x\rightarrow c}\frac{x^3-c^3}{x-c} = \lim_{x\rightarrow c}\frac{(x^2+xc+c^2)(x-c)}{x-c} = \lim_{x\rightarrow c}(x^2+xc+c^2) = 3c^2.$$
        
        \item[b)]
        
            Let $f(x) = 1/x$. Then $$f'(c) = \lim_{x\rightarrow c}\frac{(1/x)-(1/c)}{x-c} = \lim_{x\rightarrow c}\frac{\frac{c-x}{xc}}{x-c} = \lim_{x\rightarrow c}\frac{1}{xc} = \frac{1}{c^2}.$$
        
    \end{itemize}
    \item[2.]
    
        Let $f(x) = x^{1/3}$. Then $f'(0) = \lim_{x\rightarrow 0}\frac{x^{1/3} - 0}{x - 0} = \lim_{x\rightarrow 0}\frac{1}{x^{2/3}}$ which does not exist. Therefore $f'(x)$ does not exist at $x=0$.
    
    \item[4.]
    
        Let $f(x) = x^2$ if $x$ is rational and $f(x) = 0$ if $x$ is irrational. Then, $f'(0) = \frac{f(x)-f(0)}{x-0} = \frac{f(x}{x} = \lim_{x\rightarrow 0}g(x)$ where $g(x) = x$ if $x$ is rational and $g(x) = 0$ if $x$ is irrational. So $f$ is differentiable at $0$ if and only if $\lim_{x\rightarrow 0}g(x)$ exists. However, by the squeeze theorem on $-|x| \leq g(x) \leq |x|$, $\lim_{x\rightarrow 0}g(x) = 0$ so $f'(0) = 0$.
    
    \item[5.]
    \begin{itemize}
        \item[a)]
        
            Let $f(x) = \frac{x}{1+x^2}$. Then, by the quotient rule, $$f'(x) = \frac{(1)(1+x^2) - (x)(2x)}{(1+x^2)^2} = \frac{1-x^2}{(1+x^2)^2}.$$
        \item[b)]
        
            Let $f(x) = \sqrt{5 - 2x + x^2}$. Then, by the chain rule, $$f'(x) = \frac{1}{2\sqrt{5-2x+x^2}}(2x-2) = \frac{x - 1}{\sqrt{5-2x+x^2}}.$$
        
        \item[c)]
        
            Let $f(x) = (\sin(x^k))^m$. Then, by double chain rule we have $$f'(x) = m(\sin(x^k))^{m-1}\cos(x^k)(kx^{k-1}) = mkx^{k-1}(\sin(x^k))^{m-1}\cos(x^k) $$
        
        \item[d)]
        
            Let $f(x) = \tan(x^2)$. Then, by the chain rule, we have $$f'(x) = \sec^2(x^2)2x.$$
        
    \end{itemize}
    
    \item[9.]
    
        Let $f(-x) = f(x)$ for all $x$. Then we have $$f'(-c) = \lim_{x \rightarrow -c}\frac{f(x) - f(-c)}{x-(-c)} = \lim_{x\rightarrow c}\frac{f(-x) - f(-c)}{-x + c} = \lim_{x\rightarrow c}\frac{f(x)-f(c)}{-(x-c)} = -f'(c).$$
    
        Let $f(-x) = -f(x)$ for all $x$. Then we have $$f'(-c) = \lim_{x \rightarrow -c}\frac{f(x) - f(-c)}{x-(-c)} = \lim_{x\rightarrow c}\frac{f(-x) - f(-c)}{-x + c} = \lim_{x\rightarrow c}\frac{-(f(x)-f(c))}{-(x-c)} = f'(c).$$
    
    \item[11.] Assume $L'(x) = 1/x$.
    \begin{itemize}
        \item[a)]
        
            $(L(2x+3))' = L'(2x+3)(2) = \frac{2}{2x+3}.$
        
        \item[b)]
        
            $((L(x^2))^3)' = 3(L(x^2))^2L'(x^2)2x. = \frac{6(L(x^2))^2}{x}.$
        
        \item[c)]
        
            $(L(ax))' = L'(ax)(ax)' = \frac{a}{ax} = \frac{1}{x}.$
        
        \item[d)]
        
            $(L(L(x)))' = L'(L(x))L'(x)(x') = \frac{1}{xL(x)}.$
        
    \end{itemize}
    
    \item[13.]
    
        $$f'(c) = \lim_{h \rightarrow 0}\frac{f(c+h)-f(c)}{h} = \lim(n(f(c+1/n)-f(c)))$$ since if we substitute $h = 1/n$ then $h\rightarrow 0 \Leftrightarrow n\rightarrow\infty$.
        
        Let $f(x) = |x|$. Then $\lim(n(f(1/n) - 0)) = \lim (1) = 1$. But $f'(0)$ does not exist. Therefore the convergence of this sequence does not imply the existence of the derivative.
    
    \item[15.]
    
        Let $f(x) = \cos(x)$ on the interval $[0,\pi]$. Then $f^{-1}(x) = \cos^{-1}(x)$ on the interval $[-1,1]$. By the inverse rule we have $$(f^{-1}(x))' = (\cos^{-1}(x))' = \frac{1}{\cos'(\cos^{-1}(x))} = \frac{-1}{\sin(\cos^{-1}(x))}.$$ Then, since $\sin(x) = \sqrt{1 - \cos^2(x)}$ on the interval $[0,\pi]$, $\sin(\cos^{-1}(x)) = \sqrt{1-x^2}$. Therefore, $$(f^{-1}(x)) = \frac{-1}{\sqrt{1-x^2}}.$$ Since this is only defined on $(-1,1)$ in the domain, $(\arccos{x})'$ is not defined on $-1$ or $1$.
    
\end{itemize}

\subsection{The Mean Value Theorem}

\begin{itemize}
    \item[2.]
    \begin{itemize}
        \item[a)]
        
            First, note that $I= (-\infty,0)\cup(0,\infty)$ has no endpoints. Let $f(x) = x + 1/x$. Then, $f'(x) = 1-1/x^2 = 0$ at $x = \pm 1$. If $|x| < 1$, then $f'(x) < 0$ so $f(x)$ is decreasing on $x\in (-1,0)\cup(0,1)$. If $|x|>1$, then $f'(x)>1$ so $f(x)$ is increasing on $x\in(-\infty,-1)\cup(1,\infty)$.
        
        \item[b)]
        
            Let $f(x) = \frac{x}{x^2+1}$. Then $f'(x) = \frac{1-x^2}{(1+x^2)^2}$. Since the denominator is always greater than or equal to 1, the derivative of $f$ always exists so the interval has no endpoints. Also, $f'(x) = \frac{1-x^2}{(1+x^2)^2} = 0$ only when $x=\pm 1$. Since the denominator is always positive, the sign of the derivative is strictly equal to the sign on the numerator. If $|x|<1$, then $f'(x) > 0$ and if $|x|>1$, then $f'(x) < 0$. Therefore $f$ is increasing on $(-1,1)$ and decreasing on $(-\infty,-1)\cup(1,\infty)$.
        
    \end{itemize}
    
    \item[3.]
    \begin{itemize}
        \item[a)]
        
            Let $f(x) = |x^2-1|$ for $x\in [-4,4]$. We can write this as the piecewise function $f(x) = x^2 - 1$ on $x\in[-4,-1)\cup(1,4]$ and $f(x) = 1-x^2$ on $x\in[-1,1]$. Therefore $f'(x)$ is defined on the interior of all of these intervals as the derivative of the respective piecewise. Then, for $-4< x<-1$, $f'(x) < 0$ and $-1<x<0$ implies that $f'(x) > 0$ gives us that $x=-1$ is a local minimum. Also, $0<x<1$ implies that $f'(x) < 0$ so $x=0$ is local max. Finally, $1<x<4$ implies that $f'(x) > 0$ and gives us that $x=1$ is a local minimum. Finally, our endpoints $x=-4$ and $x=4$ are local maximums because the nhood greater (less) than is decreasing (increasing) respectively.
        
        \item[b)]
        
            Let $f(x) = 1 - (x-1)^{2/3}$ for $x\in [0,2]$. Then $f'(x) = -\frac{2}{3}(x-1)^{-1/3}$. So $f$ is increasing on $x\in (0,1)$ and decreasing on $x\in(1,2)$. Thus our extrema points are $x=1$ and our endpoints $x=0$ and $x=2$.
        
    \end{itemize}
    
    \item[6.]
    
        Let $f(x) = \sin x$. Then, by the mean value theorem we have $\sin(x)-\sin(y) = \cos(z)(x-y)$ for all $x,y\in\mathbb{R}$ and some $z \in (x,y)$ since $\sin$ is defined on all $\mathbb{R}$. This implies that $| \sin(x) - \sin(y) | = |\cos(z)||x-y| \leq |x-y|$. So $| \sin(x) - \sin(y) | \leq |x-y|$.
    
    \item[7.]
    
        Let $f(x) = \ln(x) - \frac{x-1}{x}$. Then $f'(x) = 1/x - 1/x^2 = \frac{x-1}{x^2}$ which is greater than 0 for $x>1$. Therefore, since $f(0) = 0$, $\ln(x) < \frac{x-1}{x}$ for $x>1$. Let $h(x) = \ln(x)$, then by the mean value theorem for $x>1$, $(\ln(x)-\ln(1)) = 1/z(x-1)$ for some $z>1$. Since $1/z < 1$ and $\ln(1)=0$, we have $\ln(x) < x-1$. Combining these two inequalities we get $\frac{x-1}{x} < \ln(x) < x-1$.
    
    \item[9.]
    
        Let $f(x) = 2x^4 + x^4\sin(1/x) = x^4(2 + \sin(1/x)).$ for $x\not=0$ and $f(0)= 0.$ Then, since $\sin(c) \in [-1,1]$, $2+\sin(1/x)\geq1$ for any $x$. Also, $x^4\geq 0$ for all $x$ due to positivity of squares. Therefore $f(x) \geq 0$ for all $x$, so $f$ has an absolute minimum at $x=0$ since $f(0)= 0$. However, since $f'(x) = x^2 (-\cos(1/x) + 4 x (2 + \sin(1/x)))$ sign is based strictly on the sign of $(-\cos(1/x) + 4 x (2 + \sin(1/x)))$. Let $|x| < 1/12.$ Then $0<4x(2+\sin(1/x)) < 1$. Thus we have $\cos(1/x) = 1 \Rightarrow f'(x) < 0$ and $\cos(1/x) = 0 \Rightarrow f'(x) > 0$. Then $\cos(1/x) = 1$ for $1/x = 0 + 2\pi n$ or $x = \frac{1}{2\pi n}$ (for all $n\in\mathbb{N}$) which by archimedes principal is both on the positive and negative side of any neighborhood of 0. Similarly we have $\cos(1/x) = 0$ for $x = \frac{1}{\pi/2 + 2\pi n}$ (for all $n\in\mathbb{N}$). Therefore, $f'(x)$ is both positive and negative in every nhood of 0.
    
    \item[10.]
    
        
    
    \item[12.]
    
        
    
    \item[13.]
    
        
    
    \item[17.]
    
        
    
\end{itemize}

\subsection{L'Hospital's Rules}

\begin{itemize}
    \item[1.]
    
        
    
    \item[2.]
    
        
    
    \item[4.]
    
        
    
    \item[6.]
    
        
    
    \item[7.] a,b
    
        
    
    \item[8.] a,b
    
        
    
    \item[9.] a,b
    
        
    
    \item[13.]
    
        
    
    \item[14.]
    
        
    
\end{itemize}

\subsection{Taylor's Theorem}

\begin{itemize}
    \item[1.]
    
        
    
    \item[2.]
    
        
    
    \item[4.]
    
        
    
    \item[5.]
    
        
    
    \item[7.]
    
        
    
    \item[8.]
    
        
    
    \item[12.]
    
        
    
    \item[14.] a,b
    
        
    
    \item[19.]
    
        
    
    \item[20.]
    
        
    
    \item[23.]
    
        
    
\end{itemize}

\section{The Riemann Integral}
\subsection{Reimann Integral}
\subsection{Riemann Integrable Function}
\subsection{The Fundamental Theorem}
\subsection{The Darboux Integral}
\subsection{Approximate Integration}


\end{document}
