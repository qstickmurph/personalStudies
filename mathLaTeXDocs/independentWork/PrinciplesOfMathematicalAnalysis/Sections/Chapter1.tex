\documentclass[Main.tex]{subfiles}

\begin{document}

All numbers mentioned are real

\begin{enumerate}
    \item \textbf{If $r$ is rational and $r\not = 0$ and $x$ is irrational, prove $r+x$ and $rx$ are irrational.}
    
        
    
    \item \textbf{Prove that there is no rational number whose square is $12$}
    
        
    
    \item \textbf{Prove Proposition $1.15$}
    \begin{enumerate}
        \item \textbf{If $x\not = 0$ and $xy=xz$ then $y=z$}
        \item \textbf{If $x\not = 0$ and $xy=x$ then $y=$}
        \item \textbf{If $x\not = 0$ and $xy=1$ then $y=1/x$}
        \item \textbf{If $x\not = 0$ then $1/(1/x)=x$}
    \end{enumerate}
    \textit{This proof is similar to Proposition 1.14}
    
    \begin{enumerate}
        \item 
        \item
        \item
        \item
    \end{enumerate}
    
    \item \textbf{Let $E$ be a nonempty subset of an ordered set; suppose $\alpha$ is a lower bound of $E$ and $\beta$ is an upper bound of $E$. Prove $\alpha\leq\beta$}
    
        
    
    \item \textbf{Let $A$ be a nonempty set of real numbers which is bounded below. Let $-A$ be the set of all numbers $-x$, where $x\in A$. Prove that}
    $$\inf A = -\sup(-A)$$
    
        
    
    \item \textbf{Fix $b > 1$.}
    \begin{enumerate}
        \item \textbf{If $m,n,p,q$ are integers, $n>0,q>0$, and $r=m/m=p/q$, prove that}
        $$(b^m)^{1/n}=(b^p)^{1/q}$$
        \textbf{when $r$ is rational. Hence it makes sense to define $b^r=(b^m)^{1/n}$}
        
            
        
        \item \textbf{Prove that $b^{r+s}=b^rb^s$ if $r$ and $s$ are rational}
        
            
        
        \item \textbf{If $x$ is real, define $B(x)$ to be the set of all numbers $b^t$, where $t$ is rational and $t\leq x$. Prove that}
        $$b^r=\sup B(r)$$
        \textbf{when $r$ is rational. Hence it makes sense to define}
        $$b^x=\sup B(x)$$
        \textbf{for every real} $x$
        
            
        
        \item \textbf{Prove that $b^{x+y}=b^xb^y$ for all real $x$ and $y$}
        
            
        
    \end{enumerate}
    
    \item \textbf{Fix $b>1,y>0,$ and prove that there is a unique real $x$ such that $b^x=y$, by completing the following outline. (This $x$ is called the logarithm of $y$ to the base $b$.)}
    \begin{enumerate}
        \item \textbf{For any positive integer $n$, $b^n-1\geq n(b-1)$}
        
            
        
        \item \textbf{Hence $b-1\geq n(b^{1/n}-1)$}
        
            
        
        \item \textbf{If $t>1$ and $n>(b-1)/(t-1)$ then $b^{1/n}<t$}
        
            
        
        \item \textbf{If $w$ issuch that $b^w<y$, then $b^{w+(1/n)}<y$ for sufficiently large $n$; to see this, apply part (c) with $t=y\cdot b^{-w}$}
        
            
        
        \item \textbf{If $b^w>y$, then $b^{w-1/n}>y$ for sufficiently large $n$}
        
            
        
        \item \textbf{Let $A$ be the set of all $w$ such that $b^w<y$, and show that $x=\sup A$ satisfies $b^x=y$}
        
            
        
        \item \textbf{Prove that this $x$ is unique}
        
            
        
    \end{enumerate}
    
    \item \textbf{Prove that no order can be defined in the complex field that turns it into an ordered field. Hint: $-1$ is a square}
    
        
    
    \item \textbf{Suppose $z=a+bi, w=c+di$. Define $z<w$ if $a<c$, or $a=c$ and $b<d$. Prove that this turns the set of all complex numbers into an ordered set. Does this ordered set have the least-upper-bound property?} \textit{This type of order is called a dictionary or lexicographic order}.
    
        
    
    \item \textbf{Suppose $z=a+bi,w=u+vi$, and}
    \begin{align*}
        a=\left( \frac{\lvert w\rvert + u}{2}\right)^{1/2}, && b=\left( \frac{\lvert w\rvert - u}{2}\right)^{1/2}.
    \end{align*}
    \textbf{Prove that $z^2=w$ if $v\geq 0$ and that ($\Bar{z})^2=w$ if $v\leq 0$. Conclude that every complex number (with one exception!) has two complex square roots.}
    
        
    
    \item \textbf{If $z$ is a complex number, prove that there exists an $r\geq 0$ and a complex number $w$ with $\lvert w\rvert=1$ such that $z=rw$. Are $w$ and $r$ always uniquely determined by $z$}
    
        
    
    \item \textbf{If $z_1,\dots ,z_n$ are complex, prove that}
    \begin{align*}
        \lvert z_1+z_2+\dots +z_n\rvert \leq \lvert z_1\rvert + \lvert z_2 \rvert + \dots + \lvert z_n \rvert 
    \end{align*}
    
        
    
    \item \textbf{If $x,y$ are complex, prove that }
    \begin{align*}
        \lvert\lvert x\rvert -\lvert y\rvert\rvert \leq \lvert x-y\rvert
    \end{align*}
    
        
    
    \item \textbf{If $z$ is a complex number such that $\lvert z\rvert =1$, that is, such that $z\Bar{z}=1$, compute}
    \begin{align*}
        \lvert 1 + z \rvert^2 + \lvert 1-z\rvert^2
    \end{align*}
    
        
    
    \item \textbf{Under what conditions does equality hold in the Schwarz inequality?}
    
        
    
    \item \textbf{Suppose $k\geq 3, \boldsymbol{x,y}\in\mathbb{R}^k$, $\lvert\boldsymbol{x-y}\rvert = d>0$, and $r>0$. Prove}
    \begin{enumerate}
        \item \textbf{If $2r>d$, there are infinitely many $\boldsymbol{z}\in\mathbb{R}^k$ such that}
        \begin{align*}
            \lvert\boldsymbol{z-x}\rvert = \lvert\boldsymbol{z-y}\rvert = \boldsymbol{r}
        \end{align*}
        \item \textbf{If $2r=d$ there is exactly one such $\boldsymbol{z}$}.
        \item \textbf{If $2r<d$, there is no such $\boldsymbol{z}$}.
    \end{enumerate}
    
    \begin{enumerate}
        \item 
        \item
        \item
    \end{enumerate}
    
    \item \textbf{Prove that}
    \begin{align*}
        \lvert\boldsymbol{x+y} \rvert^2 + \lvert\boldsymbol{x-y} \rvert^2=2\lvert\boldsymbol{x}\rvert^2+2\lvert\boldsymbol{y}\rvert^2
    \end{align*}
    \textbf{if $\boldsymbol{x}\in\mathbb{R}^k$ and $\boldsymbol{y}\in\mathbb{R}^k$. Interpret this geometrically, as a statement about parallelograms}
    
        
    
    \item \textbf{If $k\geq 2$ and $\boldsymbol{x}\in\mathbb{R}^k$, prove that there exists $\boldsymbol{y}\in\mathbb{R}^k$ such that $\boldsymbol{y}\not = 0$ but $\boldsymbol{x}\cdot\boldsymbol{y}=0$ Is this also true if $k=1$.}
    
        
    
    \item \textbf{Suppose $\boldsymbol{a,b}\in\mathbb{R}^k$. Find $\boldsymbol{c}\in\mathbb{R}^k$ and $r>0$ such that}
    \begin{align*}
        \lvert\boldsymbol{x-a}\rvert = 2\lvert\boldsymbol{x-b}\rvert
    \end{align*}
    \textbf{if and only if $\lvert\boldsymbol{x-c}\rvert = r$}
    
        
    
    \item \textbf{With reference to the Appendix, suppose that property (III) were omitted from the definition of a cut. Keep the same definitions of order and addition. Show that the resulting ordered set has the least-upper-bound property, that addition satisfies axioms (A1) through (A4) (with a different zero element) but that (A5) fails}
    
        
    
\end{enumerate}

\end{document}
