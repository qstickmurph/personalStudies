\documentclass[a4paper, 11pt]{article}
\usepackage{comment} % enables the use of multi-line comments (\ifx \fi) 
\usepackage{fullpage} % changes the margin
\usepackage[T1]{fontenc}
\usepackage{selinput}
\usepackage{enumitem}
\usepackage{listings}
\usepackage{amsmath}
\usepackage{amssymb}
\usepackage{setspace}

\SelectInputMappings{
   aacute={á},
   ntilde={ñ}
}
\newcommand{\Mod}[1]{\ (\mathrm{mod}\ #1)}

\begin{document}
%Header-Make sure you update this information!!!!
\noindent
\large\textbf{Mathematical Foundations of Cryptography} \hfill \textbf{Quinn Murphey} \\
\normalsize MAT 4353.001 \hfill Date: 02/15/19 \\
Dr. Dueñez \hfill Due Date: 02/18/19 \\
\noindent\makebox[\linewidth]{\rule{\paperwidth}{0.4pt}}
\section*{Week 4}
    2.3, 2.5, 2.6, 2.7, 2.8, 2.9, 2.12, 2.15.
    
\section*{Mandatory}

\noindent\textbf{Problem 2.3:}
    
    Let $g$ be a primitive root for $\mathbb{F}_p$
    \begin{enumerate}[label=(\alph*)]
        \item Suppose that $x=a$ and $x=b$ are both integer solutions to the congruence  $g^x \equiv h \Mod{p}$. Prove that $a\equiv b \Mod{p-1}$. Explain why this implies that the map below is well-defined.
        $$log_g:\mathbb{F}_p^* \rightarrow \frac{\mathbb{Z}}{(p-1)\mathbb{Z}}$$
        
        To prove that this function is well defined, let $a,a'\in\mathbb{F}_p$ and let $a\equiv a' \Mod{p}$. We need to show that these two elements map to the same element in $\mathbb{Z}/(p-1)\mathbb{Z}$. 
        $a=a'+ np$ for some $n\in\mathbb{Z}$.
        $$g^a=g^{a'+ np}=g^{a'}g^{np}=g^{a'}(1)=g^{a'}$$
        Therefore $log_g$ is well defined.
        
        \item Prove that $log_g(h_1h_2) = log_g(h_1) + log_g(h_2) \quad\text{ for all } h_1,h_2\in\mathbb{F}_p^*$
        
        if $log_g(h_1)= x_1 $ and $log_g(h_2)=x_2$ then $h_1=g^{x_1}$ and $h_2=g^{x_2}$ and $$h_1h_2 = g^{x_1}g^{x_2} = g^{x_1+x_2}$$ then $$log_g(h_1h_2) = x_1+x_2 = log_g(h_1)+log_g(h_2).$$
        so $log_g(h_1h_2) = log_g(h_1)+log_g(h_2)$
        
        \item Prove that $log_g(h_1^n) = n*log_g(h_1) \quad\text{ for all } h_1\in\mathbb{F}_p^*$ and $n\in\mathbb{Z}$
        
        We will prove this by induction on $n$.
        
        Base Case: 
        
    \end{enumerate}
    
\noindent\textbf{Problem 2.5:}
    
    
    
    
    
\noindent\textbf{Problem 2.6:}
    
    
    
    
    
\noindent\textbf{Problem 2.7:}
    
    
    
    
    
\noindent\textbf{Problem 2.8:}
    
    
    
    
    
\noindent\textbf{Problem 2.9:}
    
    
    
    
    
\noindent\textbf{Problem 2.12:}
    
    
    
    
    
\noindent\textbf{Problem 2.15:}
    
    
    
\end{document}
