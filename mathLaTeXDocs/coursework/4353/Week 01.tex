\documentclass[a4paper, 11pt]{article}
\usepackage{comment} % enables the use of multi-line comments (\ifx \fi) 
\usepackage{lipsum} %This package just generates Lorem Ipsum filler text. 
\usepackage{fullpage} % changes the margin
\usepackage[T1]{fontenc}
\usepackage{selinput}
\usepackage{enumitem}
\usepackage{listings}
\usepackage{amsmath}
\usepackage{amssymb}

\SelectInputMappings{
   aacute={á},
   ntilde={ñ}
}

\begin{document}
%Header-Make sure you update this information!!!!
\noindent
\large\textbf{Math Fdn. of Cryptography} \hfill \textbf{Quinn Murphey} \\
\normalsize MAT 4353.001 \hfill Date: 01/16/19 \\
Dr. Dueñez \hfill Due Date: 01/22/19 \\
\noindent\makebox[\linewidth]{\rule{\paperwidth}{0.4pt}}
\section*{Week 1}
1.1, 1.2, 1.3, 1.6, 1.9, 1.10, 1.11. Optional (extra credit): 1.4, 1.5, 1.13.
\\
\section*{Mandatory}

\textbf{1.1.} 
\begin{enumerate}[label=(\alph*)]
    \item L ALRP ZQ STDEZCJ TD HZCES L GZWFXP ZQ WZRTN
    \item THERE ARE NO SECRETS BETTER THAN THE SECRETS THAT EVERYBODY GUESSES
    \item WHEN ANGRY COUNT TEN BEFORE YOU SPEAK IF VERY ANGRY AN HUNDRED
\end{enumerate}

\noindent All three were solved using variations on the code below (which was used specifically for (c) )

\begin{lstlisting}
ciphertext = 'XJHRFTNZHMZGAHIUETXZJNBWNUTRHEPOMDNBJMAUGORFAOIZOCC'
index = 0

for i in ciphertext:
    index = index + 1
    print(chr((ord(i) - 65 - index) % 26 + 65))
\end{lstlisting} 

\noindent \textbf{1.2}
\begin{enumerate}[label=(\alph*)]
    \item I THINK THAT I SHALL NEVER SEE A BILLBOARD LOVELY AS A TREE
    \item LOVE IS NOT LOVE WHICH ALTERS WHEN IT ALTERATION FINDS
    \item IN BAITING A MOUSE TRAP WITH CHEESE ALWAYS LEAVE ROOM FOR THE MOUSE
\end{enumerate}
Same code as above used through trial and error

\noindent \textbf{1.3}
\begin{enumerate}[label=(\alph*)]
    \item IBXFEPAQLBQAAXWQWIBXFSVAXW
    \item Top row is ciphertext char and bottom row is plaintext translated char\\
        \begin{tabular}{ |c|c|c|c|c|c|c|c|c|c|c|c|c|c|c|c|c|c|c|c|c|c|c|c|c|c| } 
             \hline
                 a & b & c & d & e & f & g & h & i & j & k & l & m & n & o & p & q & r & s & t & u & v & w & x & y & z \\ 
             \hline
                 a & c & e & g & i & k & m & o & q & s & u & w & y & b & d & f & h & j & l & n & p & r & t & v & w & z \\ 
            \hline
        \end{tabular}\\
    \item  scvwv srvqz lwwgi rahww girap hwc 
\end{enumerate}

\newpage

\noindent \textbf{1.6}
\begin{enumerate}[label=(\alph*)]
    \item $a \mid b \Rightarrow \exists n \in\mathbb{Z}\left( an = b \right)$ and $b \mid c \Rightarrow \exists m \in\mathbb{Z}\left( bm = c \right)$ \\
        Therefore, $\exists n,m \in\mathbb{Z} \left( amn = c\right)$ and since $nm \in\mathbb{Z}$, $a \mid c$
    \item $a \mid b \Rightarrow \exists n \in\mathbb{Z}\left( an = b \right)$ and $b \mid a \Rightarrow \exists m \in\mathbb{Z}\left( bm = a \right)$ \\
     Therefore, $an=b$ and $bm=a$ so we can substitute the first equation into the second to obtain $anm=a$. From here we can use the law of cancellation to get $nm=1$ and since $n,m \in\mathbb{Z}$, $n=m=1$ or $n=m=-1$  it follows that $an = \pm a$. When you substitute this into the original equation $an=b$, you get $a = \pm b$
    \item This one follows immediately from the distributive property of integers: \\ $\forall n,a,b \in\mathbb{Z} , n(a+b) = (na+nb)$ which also applies to the subtraction case. Therefore, if a divides both b and c, then a divides the sum of b and c and the difference of b and c
\end{enumerate}\\

\noindent \textbf{1.9}
\begin{enumerate}[label=(\alph*)]
    \item gcd(291,252)\\
        $$291 = 252*1 + 39$$
        $$252 = 39*6 + 18$$
        $$39 = 18*2 + 3$$
        $$18 = 3*6 + 0$$
        $$\therefore gcd(291,252)=3$$
    \item gcd(16261,85652)\\
        $$85652 = 16261*5 + 4347$$
        $$16261 = 4347*3 + 3220$$
        $$4347 = 3220*1 + 1127$$
        $$3220 = 1127*2 + 966$$
        $$1127 = 966*1 + 161$$
        $$966 = 161*6 + 0$$
        $$\therefore gcd(16261,85652)=161$$
    \item gcd(139024789,93278890)
        $$139024789 = 93278890*1 + 45745899$$
        $$93278890 = 45745899*2 + 1787092$$
        $$45745899 = 1787092*25 + 1068599$$
        $$1787092 = 1068599*1 + 718493$$
        $$1068599 = 718493*1 + 350106$$
        $$718493 = 350106*2 + 18281$$
        $$350106 = 18281*19 + 2767$$
        $$18281 = 2767*6 + 1679$$
        $$2767 = 1679*1 + 1088$$
        $$1679 = 1088*1 + 591$$
        $$1088 = 591*1 + 497$$
        $$591 = 497*1 + 94$$
        $$497 = 94*5 + 27$$
        $$94 = 27*3 + 13$$
        $$27 = 13*2 + 1$$
        $$13 = 1*13 + 0$$
        $$\therefore gcd(139024789,93278890)=1$$
    \item gcd(16534528044,8332745927)
        $$16534528044 = 8332745927*1 + 8201782117$$
        $$8332745927 = 8201782117*1 + 130963810$$
        $$8201782117 = 130963810*62 + 82025897$$
        $$130963810 = 82025897*1 + 48937913$$
        $$82025897 = 48937913*1 + 33087984$$
        $$48937913 = 33087984*1 + 15849929$$
        $$33087984 = 15849929*2 + 1388126$$
        $$15849929 = 1388126*11 + 580543$$
        $$1388126 = 580543*2 + 227040$$
        $$580543 = 227040*2 + 126463$$
        $$227040 = 126463*1 + 100577$$
        $$126463 = 100577*1 + 25886$$
        $$100577 = 25886*3 + 22919$$
        $$25886 = 22919*1 + 2967$$
        $$22919 = 2967*7 + 2150$$
        $$2967 = 2150*1 + 817$$
        $$2150 = 817*2 + 516$$
        $$817 = 516*1 + 301$$
        $$516 = 301*1 + 215$$
        $$301 = 215*1 + 86$$
        $$215 = 86*2 + 43$$
        $$86 = 43*2 + 0$$
        $$\therefore gcd(16534528044,8332745927)=43$$
\end{enumerate}

\noindent \textbf{1.10}
\begin{enumerate}[label=(\alph*)]
    \item 291*13 + 252*(-15) = 3
    \item 16261*(-79) + 85652*15 = 161
    \item 139024789*6944509 + 93278890*(-10350240) = 1
    \item 16534528044*81440996 + 8332745927*(-161602003) = 43
\end{enumerate}

\noindent \textbf{1.11}
\begin{enumerate}[label=(\alph*)]
    \item Suppose that there are integers $u$ and $v$ satisfying $au + bv = 1$. Prove that $gcd(u, v) = 1$.\\
    
        Assume that $u$ and $v$ have a common factor $n$ s.t. $u=zn$ and $v=yn$. Then, by the distributive law, we have $au + bv = n\left( az+by \right) = 1$. Since we know that $az$ and $by$ are Integers, we then know that $(az + by)$ is an Integer. Therefore we must have an $n$ s.t. $az+by = \frac{1}{n}$ which can only be true for $n=\pm 1$. Therefore, the greatest common factor of $u$ and $v$ is 1
    \item Suppose that there are integers u and v satisfying au + bv = 6. Is it necessarily true that gcd(a, b) = 6? If not, give a specific counterexample, and describe in general all of the possible values of gcd(a,b)?\\
    
        Using the same reasoning as (a). Assume that $u$ and $v$ have a common factor $n$ s.t. $u=zn$ and $v=yn$.  Then $n\left( az+by \right) = 6$ and $\left( az+by \right) = \frac{6}{n}$ which is an integer for all $n \in \{$integer factors of 6$\}$. Therefore gcd($u$,$v$)$\in$\{integer factors of 6\}
    \item Suppose that ($u_1,v_1$) and ($u_2,v_2$) are two solutions in integers to the equation $au+bv = 1$. Prove that $a$ divides $v_2 - v_1$ and that $b$ divides $u_2 - u_1$.\\
    
        Since $au_1 + bv_1 = au_2 + bv_2$ with some algebra we get $u_2 - u_1 = (bv_1 - bv_2)/a$. Since $u_2 - u_1$ is an Integer, so is $(bv_1 - bv_2)/a$ proving that $(bv_1 - bv_2)$ is divisible by a. Since we know gcd($a,b$)=1, b is not divisible by 1. Meaning that $v_1 - v_2$ must be divisible by a.
        A similar proof can be used to show that $b$ divides $u_2 - u_1$
    \item More generally, let $g$ = gcd($a,b$) and let ($u_0,v_0$) be a solution in integers to $au + bv = g$. Prove that every other solution has the form $u = u_0 + kb/g$ and $v = v_0 − ka/g$ for some integer $k$. (This is the second part of Theorem 1.11.)\\
    
        Given that $au_0 + bv_0 = g$, we can verify directly that $u = u_0 + kb/g$ and $v = v_0 − ka/g$ are always alternate solutions to the equation $au+bv=g$ for $k \in\mathbb{Z}$ by substitution.
        $$au+bv=g$$
        $$\Rightarrow a \left( u_0+\frac{kb}{g}\right)+ b \left( v_0 -\frac{ka}{g}\right)=g$$
        $$\Rightarrow \left( au_0 + bv_0\right) + \left(\frac{kab}{g} - \frac{kab}{g}\right) =g$$
        $$\Rightarrow \left( g\right)+ \left( 0\right) = g$$
        Which proves that for all $k \in\mathbb{Z}$, $u  = u_0 + kb/g$ and $v = v_0 - ka/g$ satisfy the equation $au+bv=g$
\end{enumerate}
\end{document}
