\documentclass[letter paper, 12pt]{article}
\usepackage{comment} % enables the use of multi-line comments (\ifx \fi) 
\usepackage{fullpage} % changes the margin
\usepackage[T1]{fontenc}
\usepackage{selinput}
\usepackage{enumitem}
\usepackage{listings}
\usepackage{amsmath}
\usepackage{amssymb}
\usepackage{setspace}

\newcommand{\Mod}[1]{\ (\mathrm{mod}\ #1)}

\begin{document}
%Header-Make sure you update this information!!!!
\noindent
\large\textbf{Modern Abstract Algebra II} \hfill \textbf{Quinn Murphey} \\
\normalsize MAT 4953.001 \hfill Date: 03/30/19 \\
Dr. Patterson \hfill Due Date: 04/01/19 \\
\noindent\makebox[\linewidth]{\rule{\paperwidth}{0.4pt}}

\section*{Problem Set 7}
\doublespacing
\noindent\textbf{Problem 1:}
    
    Give a nice proof for the second problem from Board Meeting 7, which states that if $V$ is a vector space over a field $F$, and $S$ and $L$ are finite sets of vectors such that $S$ spans $V$ over $F$ and $L$ is linearly independent over $F$, then $\lvert S\rvert \geq \lvert L\rvert$. If you're feeling bold, you can try to solve this problem using only the hint $I$ provided this week: induct on the size of $S$.\\
    
    We will prove this by induction on $n$ which will be the cardinality of $S$ and subsequently the dimension of $V$. 
    
    Base Case: We will start with a spanning set of size $1$ because the empty set is linearly dependent. Let $n=1$, then every element in $V$ is a linear combination of the one element of $S$ which we'll call $s_1 \not= 0$. Now, assume that $L$ has greater than 1 element. Then $l_1,l_2\in L$ can be represented 
    $$l_1 = c_1s_1 \qquad l_2 = c_2s_1$$
    then we have
    $$c_2l_1 - c_1l_2 = 0.$$
    
    Induction Step: Now assume the theorem is true for $n=k$, then $L$ has more than $k$ elements so we'll say $k+1$. Each $l_i\in L$ can be written
    $$l_i = \sum_{j=1}^{k}c_{i,j}s_j.$$
    We can assume that for $j$, not all $c_{1,j} = 0$ because that would contradict our induction hypothesis. Multiply each element $l_i$ with $c_{i,k+1} \not= 0$ by $\frac{1}{c_{i,k+1}}$ so that every new element $$l_i'=\frac{1}{c_{i,k+1}}\sum_{j=1}^{k}c_{i,j}s_j.$$ 
    Now, since every element's $k$th coefficient is equal to $1$, we can create a new set of elements $T$ where 
    $$t_i = l'_i - l'_1 \qquad \text{if }c_{i,k}\not=0 \text{ and } i \not= 1$$
    $$t_i = l'_i \qquad \text{if } c_{i,k+1}=0 $$ 
    of $k$ elements where each element's $k$th term equals $0$ so now $T$, a linearly independent set of order $k$ over a vector space spanned by $k-1$ elements which contradicts our indutive hypothesis.
    
    Therefore, it is impossible to have an independent subset of elements larger than a spanning set of the given vector space.
    
    
    Assume that $n=\lvert S\rvert$ is the cardinality of $S$ and the dimension of $V$. This means for all $v\in V$, we can represent $v$ as a linear combination of elements $s_1,s_2,\dots,s_n \in S$. Then, let there be a linearly independent set $L \subseteq V$ such that $\lvert S\rvert < \lvert L\rvert$. Then, each element $l_1,l_2,\dots,l_n,l_{n+1},\dots\in L$ can be decomposed into scalar combinations of $S$.
    
\noindent\textbf{Problem 2:}
    
    In the FLOW videos, I pointed out that if $E$ is an extension field of $F$, then we can view $E$ as a vector space over $F$. Prove the following conjecture made by John: if $F$, $H$, and $K$ are fields such that $F \subseteq H \subseteq K$, $H$ has degree $m$ over $F$, and $K$ has degree $n$ over $H$, then $K$ has degree $m\cdot n$ over $F$.
    
    Since $K$ has degree $n$ over $H$, every element $x$ in $K$ which has basis $= \{k_1, k_2, \dots, k_n\}$  can be written $$x = \sum_{j=1}^na_{j}k_j$$ for $a_{j}\in H$. However, since $H$ is a vector space with basis $=\{h_1, h_2, \dots, h_m \}$ of degree $m$ over $F$ each of those $a_{j}$ above may be written $$a_{j} = \sum_{i=1}^mb_{j,i}h_i$$ for $b_{j,i}\in F$. Then we can rewrite the first equation to be $$x = \sum_{j=1}^n\left(\sum_{i=1}^mb_{j,i}h_i\right)k_j$$ for $b_{j,i}\in F$.
    
    Then due to the independence of all $h_i$ and $k_i$ and the nature of both being a field, we can interpret every element $x\in K$ as a linear combination (scalars in $F$) of $nm$ independent elements $\{h_ik_j \mid h_i\in [H]_\mathcal{B},k_j\in [K]_\mathcal{B}\}$.
    
\noindent\textbf{Problem 3:}
    
    Consider the polynomial $p(x) = ax^2 + bx + c$ in $C[x]$, where $a\not= 0$. Suppose we want to find the roots of $p(x)$ and don't have a handy formula for doing so. The first step we often take is to multiply the polynomial by $\frac{1}{a}$ so that we have a monic polynomial (one with lead coefficient 1). The next step is the tricky one. Find $a$ substitution $u = x + k$, where $k \in C$, such that when $p(x)$ is written in terms of $u$, the resulting polynomial has no first-degree term - which means it is much easier to solve. Use this approach to derive the famous quadratic formula.
    
    Let $x = u + -b/2$ (where $b$ is the second coefficient of our new polynomial). Then, substituting in this we obtain 
    \begin{align*}
        u^2+(c-b^2/2+b^2/4) &= 0\\
        4u^2 &= b^2 - 4c\\
        2(x-b/2) &= \pm\sqrt{b^2-4c}\\
        x = \frac{b\pm\sqrt{b^2-4c}}{2}
    \end{align*}
    
    
\end{document}
