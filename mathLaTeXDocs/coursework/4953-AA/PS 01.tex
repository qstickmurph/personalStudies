\documentclass[a4paper, 11pt]{article}
\usepackage{comment} % enables the use of multi-line comments (\ifx \fi) 
\usepackage{fullpage} % changes the margin
\usepackage[T1]{fontenc}
\usepackage{selinput}
\usepackage{enumitem}
\usepackage{listings}
\usepackage{amsmath}
\usepackage{amssymb}
\usepackage{setspace}

\newcommand{\Mod}[1]{\ (\mathrm{mod}\ #1)}

\begin{document}
%Header-Make sure you update this information!!!!
\noindent
\large\textbf{Modern Abstract Algebra II} \hfill \textbf{Quinn Murphey} \\
\normalsize MAT 4953.001 \hfill Date: 01/30/19 \\
Dr. Patterson \hfill Due Date: 01/25/19 \\
\noindent\makebox[\linewidth]{\rule{\paperwidth}{0.4pt}}

\section*{Problem Set 1}
\doublespacing
\textbf{Problem 1:}
    
    Prove that $\mathbb{Z}_n$ is an integral domain if and only if n is prime. Also prove that $\mathbb{Z}_n$ is a field if and only if n is prime \\
    
    Proving that $\mathbb{Z}_n$ is a field after proving that it's an Integral Domain just involves proving that $\mathbb{Z}_n$ contains inverses for all nonzero $z \in\mathbb{Z}_n$ (or prove that $\mathbb{Z}_n$ is finite.)
    
    We will first prove the right direction. 

    Assume $n$ is prime. To prove that $\mathbb{Z}_n$ is an Integral Domain, we must show that for all nonzero $a,b \in\mathbb{Z}_n{}$ we have $a*b\not = 0$. In $\mathbb{Z}_n$ the $0$ element is the normal subgroup generated by n: $\langle n\rangle = \{ kn \mid k\in\mathbb{Z}\}$. This means that for any product of elements to be equal to $0$, they must multiply to a multiple of $n$. However since $n$ is prime, for $n$ to divide $ab$, $n\mid a$ or $n\mid b$. Which is a contradiction to our assumption that $a,b$ are nonzero. Therefore, $\mathbb{Z}_n$ is an Integral Domain. Since $\mathbb{Z}_n$ is finite, it is also a field.
    
    We will next prove the left direction. Since $\mathbb{Z}_n$ is finite. If it is an Integral Domain, then it is also a field and vice versa.
    
    So assume that $\mathbb{Z}_n$ is an Integral Domain and prove that $n$ is prime. For $\mathbb{Z}_n$ to be an integral domain we must have $\forall a,b\in\mathbb{Z}_n(a,b\not = 0 \Rightarrow a*b \not = \langle n\rangle)$. Therefore there are no numbers which multiply to be a multiple of $n$ unless one of them is a multiple of $n$. Since each element of $\mathbb{Z}_n$ is less than $n$, each element can be decomposed into a product of primes less than $n$. Since all integers $n$ can be decomposed into a product of primes less than or equal to $n$. The only such $n$ which is not a product of elements of $\mathbb{Z})_n$ is when $n$'s factors are only $n$. Therefore $n$ is a prime number. 
    
    Since we have proven both directions, this completes the proof.\\
\newpage
\noindent\textbf{Problem 2:}
    
    Let $a(x) = x^4 - x^3 +5x-6$ and let $b(x) = 2x^2 +x-4.$ Divide $a(x)$ by $b(x)$ and obtain a quotient $q(x)$ and remainder $r(x)$, with deg $r(x) <$ deg $b(x)$, $(a)$ in $Q[x]$, and $(b)$ in $Z_7[x]$. What relationship can you find between your answer to $(a)$ and your answer to $(b)$? (Wolfram Alpha can help you check your answer to at least one of these, but show your work so that I can see that you understand the process.)
    
    $$q(x) = \frac{x^2}{2}- \frac{3x}{4}-\frac{11}{8}$$
    $$r(x) = \frac{5x}{8} - \frac{1}{2}$$
    (work on attached paper)\\
    
\noindent\textbf{Problem 3:}
    
    In class you investigated the validity of the statement, "If $p(x)$ and $q(x)$ are nonzero polynomials in $R[x]$ of degree $m$ and $n$, respectively, then $p(x) · q(x)$ is a polynomial of degree $m + n$,"  various settings. The team of Angel, Chris, and Nathan conjectured, or at least suggested the conjecture, that this statement is true if and only if $R$ is an integral domain. Prove this conjecture.
    
    Polynomial multiplication is defined on the back of the Problem Set. For a polynomial to be of degree $m+n$ the highest nonzero coefficient is $c_{m+k}$. Which is equal to $\sum\limits_{i=0}^{k} a_{k-i}b_i$. We can easily show that this is nonzero since $a_m$ and $b_n$ are both nonzero so their product is nonzero. All other products will have at least one of the two equal to zero since $a_{k>m}=0$ and $b_{k>n}=0$ so the total sum just equals $a_mb_n$
    
\noindent\textbf{Problem 4:}
    
    Suppose we are given a line segment of length 1. Prove that we can construct, for any positive integer $n$, a line segment of length exactly $\sqrt{n}$ 
    
    If we construct a right triangle with legs with length 1 and $\sqrt{n-1}$ we get a hypotenuse of $\sqrt{n}$. We can do this inductively. Since we know $\sqrt{1}=1$ and we are given a line segment with length $1$ and being able to construct $\sqrt{n}$ from $\sqrt{n-1}$ we can construct all natural number square roots.
    
\section*{Problem on Back}

    $p(x) = a_0 + a_1x + a_2x^2 + \dots + a_mx^m$ and $q(x) = b_0 + b_1x + b_2x^2 +\dots +b_nx^n$. We defined multiplication of polynomials to be $p(x) * q(x) = c_0 + c_1x + c_2x^2 + \dots + c_{m+n}$ where $c_k = \sum\limits_{i=0}^{k} a_{k-i}b_i$ 
    Prove that this is the same as multiplying via the distributive property and adding like terms.
    
    (see attached work)
    
\end{document}
